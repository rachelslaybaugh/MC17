\documentclass{anstrans}
%%%%%%%%%%%%%%%%%%%%%%%%%%%%%%%%%%%
\title{Simplified P$_N$ Equations for Nonclassical Transport with Isotropic Scattering}
\author{R. Vasques, R.N. Slaybaugh}

\institute{
Department of Nuclear Engineering, University of California, Berkeley, Berkeley, CA 94720-1730}

\email{richard.vasques@fulbrightmail.org \and slaybaugh@berkeley.edu}


%%%% packages and definitions (optional)
\usepackage{graphicx} % allows inclusion of graphics
\usepackage{booktabs} % nice rules (thick lines) for tables
\usepackage{microtype} % improves typography for PDF
\usepackage{bm}
\usepackage[capitalize,compress]{cleveref}

% cleveref package
\crefalias{subequation}{equation}
\crefformat{pluraleq}{Eqs.~(#2#1#3)}
\newcommand{\creflastconjunction}{, and\nobreakspace}

\newcommand{\SN}{S$_N$}
\renewcommand{\vec}[1]{\bm{#1}} %vector is bold italic
\newcommand{\vd}{\bm{\cdot}} % slightly bold vector dot
\newcommand{\grad}{\vec{\nabla}} % gradient
\newcommand{\ud}{\mathop{}\!\mathrm{d}} % upright derivative symbol
\def\bal#1\nal{\begin{align}#1\end{align}}
\def\bala#1\nala{\begin{align*}#1\end{align*}}
\def\bsub#1\nsub{\begin{subequations}#1\end{subequations}}
\newcommand{\f}{\frac}
\newcommand{\ux}{{\bm x}}
\newcommand{\un}{{\bm n}}
\newcommand{\unab}{{\bf \nabla}}
\newcommand{\bg}{\big>}
\newcommand{\bl}{\big<}
\newcommand{\su}{\big< s\big>}
\newcommand{\sd}{\big< s^2\big>}
\newcommand{\st}{\big< s^3\big>}
\newcommand{\sq}{\big< s^4\big>}
\renewcommand{\sc}{\big< s^5\big>}
\renewcommand{\ss}{\big< s^6\big>}
\newcommand{\sue}{\big< s\big>_\epsilon}
\newcommand{\sde}{\big< s^2\big>_\epsilon}
\newcommand{\ste}{\big< s^3\big>_\epsilon}
\newcommand{\sqe}{\big< s^4\big>_\epsilon}
\newcommand{\sce}{\big< s^5\big>_\epsilon}
\newcommand{\sse}{\big< s^6\big>_\epsilon}
\newcommand{\wsa}{\widehat\Sigma_a}
\newcommand{\wst}{\widehat\Sigma_t}
\newcommand{\wq}{\widehat Q}
\newcommand{\ep}{\varepsilon}
\newcommand{\vi}{{\varphi}}
\newcommand{\uom}{{\bf \Omega}}
\newcommand{\setb}{\mathcal{B}}


\begin{document}
%%%%%%%%%%%%%%%%%%%%%%%%%%%%%%%%%%%%%%%%%%%%%%%%%%%%%%%%%%%%%%%%%%%%%%%%%%%%%%%%
\section{Introduction}

The nonclassical transport equation \cite{larvas11} was developed to address transport problems in which the particle flux is not attenuated exponentially.
It consists of a linear Boltzmann equation on an extended phase space, able to model particle transport for any given free-path distribution.

In this paper we consider the one-speed nonclassical transport equation with isotropic scattering. This equation is written as
%\cite{larvas11}
\bal\label{eq1}
&\f{\partial \Psi}{\partial s}(s) + \uom\cdot\unab\Psi(s) + \Sigma_t(s)\Psi(s) =\\
&\qquad \f{\delta(s)}{4\pi}\left[\int_{4\pi}\int_0^\infty c\Sigma_t(s')\Psi(\ux,\uom',s')ds'd\Omega' + Q(\ux)\right],\nonumber
\nal
where $s$ describes the free-path of a particle (distance traveled since the particle's previous interaction), $\Psi(s)=\Psi(\ux,\uom,s)$ is the nonclassical angular flux, $c$ is the scattering ratio (probability of scattering), and $Q$ is an isotropic source.
The classical scalar particle flux $\Phi(\ux)$ can be recovered from the solution of \cref{eq1} by integrating over the free-path $s$ and over angle:
$\Phi(\ux) = \int_{4\pi}\int_0^\infty\Psi(\ux,\uom,s)dsd\Omega$.

The total cross section $\Sigma_t$ is a function of the free-path $s$ and satisfies $p(s)= \Sigma_t(s)e^{-\int_0^s\Sigma_t(s')ds'}$, where $p(s)$ is the free-path distribution function.
For $m=0, 1, 2,...,$ we define the $m$-th raw moment of $p(s)$ as
$\bl s^m\bg = \int_0^\infty s^mp(s)ds.$
The following identity holds for $m=1,2,...$ : 
\bal\label{eq2}
\bl s^m\bg = m\int_0^\infty s^{m-1}e^{-\int_0^s\Sigma_t(s')ds'}ds.
\nal
A nonclassical diffusion approximation of \cref{eq1} has been previously derived \cite{larvas11}.
In this paper we use an asymptotic analysis to derive more accurate diffusion approximations to \cref{eq1}.
If $p(s)$ is given by an exponential (classical transport), these approximations reduce to the simplified P$_N$ (SP$_N$) equations \cite{mcl11}; therefore, they are labeled \textit{nonclassical} SP$_N$ equations.

%%%%%%%%%%%%%%%%%%%%%%%%%%%%%%%%%%%%%%%%%%%%%%%%%%%%%%%%%%%%%%%%%%%%%%%%%%%%%%%%
\section{Asymptotic Analysis}
Let us write \cref{eq1} in the mathematically equivalent form
\bala
&\f{\partial \Psi}{\partial s}(s) + \uom \cdot \unab \Psi(s) + \Sigma_t(s)\Psi(s) = 0, \qquad s>0,\\
&\Psi(0) = \f{1}{4\pi}\left[\int_{4\pi}\int_0^\infty c\Sigma_t(s')\Psi(\ux,\uom',s')ds'd\Omega' + Q(\ux)\right],
\nala
where $\Psi(0) = \displaystyle{\lim_{s\to 0^+}\Psi(s)} = \Psi(0^+)$.
Defining $0<\ep\ll 1$, we perform the following scaling: $\Sigma_t(s) = \ep^{-1}\Sigma_t(s/\ep)$; $c = 1-\ep^2\kappa$; and $Q(\ux) = \ep q(\ux)$;
where $\kappa$ and $q$ are $O(1)$.
Under this scaling,
\bal\label{eq3}
\bl s^m\bg &=
%\ep^m\int_0^\infty \left(\f{s}{\ep}\right)^m \f{1}{\ep}\Sigma_t(s/\ep)e^{-\int_0^s\f{1}{\ep}\Sigma_t(s'/\ep)ds'}ds  \\ &=
\ep^m\int_0^\infty s^m\Sigma_t(s)e^{-\int_0^s\Sigma_t(s')ds'}ds = \ep^m\bl s^m\bg_\ep,
\nal
where $\bl s^m\bg_\ep$ is $O(1)$.

Next, we define $\psi(\ux,\uom,s)\equiv\f{\ep\sue}{e^{-\int_0^s\Sigma_t(s')ds'}}\Psi(\ux,\uom,\ep s)$.
This satisfies
\bsub\label[pluraleq]{eq4}
\bal
&\f{\partial \psi}{\partial s}(s) + \ep\uom \cdot \unab \psi(s)= 0, \qquad s>0,\label{eq4a}\\
&\psi(0) = \label{eq4b}\\
&\f{1}{4\pi}\left[\int_{4\pi}\int_0^\infty (1-\ep^2\kappa)p(s')\psi(\ux,\uom',s')ds'd\Omega' + \ep^2 \sue q(\ux)\right],\nonumber
\nal
\nsub
and the classical scalar flux can be written as $\Phi(\ux) = \int_{4\pi}\int_0^\infty \psi(\ux,\uom,s)\f{e^{-\int_0^s\Sigma_t(s')ds'}}{\sue}ds d\Omega$.
Integrating \cref{eq4a} over $0<s'<s$ and using \cref{eq4b}, we obtain
\bala%\label{2.9}
&\left(I+\ep\uom\cdot\unab\int_0^s(\cdot)ds\right)\psi = \\
&\qquad\qquad \f{1}{4\pi}\left[\int_0^\infty (1-\ep^2\kappa)p(s')\vi(\ux,s')ds' + \ep^2\sue q\right],
\nala
where $\vi(\ux,s) = \int_{4\pi}\psi(\ux,\uom,s)d\Omega$.
Inverting the operator on the left-hand side of the above equation and expanding it in a power series, we obtain
\bal\label{eq5}
\psi &= \left(\sum_{n=0}^{\infty}(-\ep)^n\left(\uom\cdot\unab\int_0^s(\cdot)ds\right)^n\right) \times \\
&\qquad \qquad \left[\int_0^\infty \f{1-\ep^2\kappa}{4\pi}p(s')\vi(\ux,s')ds' + \ep^2\sue \f{q}{4\pi}\right].\nonumber
\nal
Let us define $\unab_0= \f{1}{3}\unab^2$ and $\setb =\unab_0\left(\int_0^s(\cdot)ds\right)^2$. Then, using the identity \cite{fra07}
\bala
\f{1}{4\pi}\int_{4\pi}\left(\uom\cdot\unab\int_0^s(\cdot)ds\right)^nd\Omega = 
\f{1+(-1)^n}{2}\f{3^{n/2}}{n+1}\setb^{n/2},
\nala
for $n=0,1,2,...$ , we integrate \cref{eq5} over the unit sphere and obtain
\bala
\vi &= \left(\sum_{n=0}^{\infty}\f{\ep^{2n}}{2n+1}(3\setb)^n\right)\times\\
&\qquad\qquad\left[\int_0^\infty (1-\ep^2\kappa)p(s')\vi(\ux,s')ds' + \ep^2\sue q\right].
\nala
Inverting the operator on the right-hand side of this equation and once again expanding it in a power series, we get
\bsub\label[pluraleq]{eq6}
\bal
&\left(I- \ep^2 \setb - \f{4\ep^4}{5}\setb^2 - \f{44\ep^6}{35}\setb^3 + O(\ep^8)\right)\vi =\label{eq6a}\\
&\qquad\qquad\int_0^\infty (1-\ep^2\kappa)p(s')\vi(\ux,s')ds' + \ep^2\sue q.\nonumber
\nal
The solution of this equation is
\bal\label{eq6b}
&\vi(\ux,s) = \\ 
&\qquad \left(I + \ep^2\f{s^2}{2!}\unab_0 + \f{9\ep^4}{5}\f{s^4}{4!}\unab_0^2 + \f{27\ep^6}{7}\f{s^6}{6!}\unab_0^3 + O(\ep^8)\right)\phi(\ux),\nonumber
\nal
\nsub
where $\phi(\ux) = {\sum_{n=0}^\infty  \ep^2\phi_{2n}(\ux)}$, with $\phi_{2n}(\ux)$ undetermined at this point.

We now multiply \cref{eq6} by $e^{-\int_0^s\Sigma_t(s')ds'}/\sue$ and operate on them by $\int_0^\infty (\cdot) ds$. Using \cref{eq2,eq3,eq5}, \cref{eq6b} yields an expression for the scalar flux:
\bala
\Phi(\ux) = &\left(I + \ep^2\f{\ste}{3!\sue}\unab_0 + \f{9\ep^4}{5}\f{\sce}{5!\sue}\unab_0^2 + \right.\\
&\hspace{7em} \left. \f{27\ep^6}{7}\f{\bl s^7\bg_\ep}{7!\sue}\unab_0^3 + O(\ep^8)\right)\phi(\ux). 
\nala
We can write $\int_0^\infty p(s)\vi(\ux,s)ds = \left(\sum_{n=0}^\infty\ep^{2n}U_n\unab_0^n \right)\Phi(\ux),$
with $U_0 = 1$; $U_1= \f{\sde}{2!}-\f{\ste}{3!\sue}$;
$U_2 = \f{9}{5}\left[\f{\sqe}{4!}-\f{\sce}{5!\sue}\right]-\f{\ste}{3!\sue}U_1$;
$U_3 =\f{27}{7}\left[\f{\sse}{6!}-\f{\bl s^7\bg_\ep}{7!\sue}\right]-\f{9}{5}\f{\sce}{5!\sue}U_1-\f{\ste}{3!\sue}U_2;$ ...  

\Cref{eq6a} can be rewritten as
\bal\label{eq7}
&\left(\sum_{n=0}^\infty \ep^{2n}V_n\unab_0^n\right)\Phi(\ux) = \\
&\qquad\qquad (1-\ep^2\kappa)\left(\sum_{n=0}^\infty\ep^{2n}U_n\unab_0^n \right)\Phi(\ux) + \ep^2\sue q(\ux),\nonumber
\nal
where $V_0 = 1$; $V_1= -\f{\ste}{3!\sue}V_0$; $V_2 = -\f{9}{5}\f{\sce}{5!\sue}V_0 -\f{\ste}{3!\sue}V_1$; $V_3 = -\f{27}{7}\f{\bl s^7 \bg_\ep}{7!\sue}V_0 - \f{9}{5}\f{\sce}{5!\sue}V_1-\f{\ste}{3!\sue}V_2$; ... 

Finally, rearranging the terms in \cref{eq7} we get
\bal\label{eq8}
\left(\sum_{n=0}^\infty \ep^{2n}\left[W_{n+1}\unab_0^{n+1}+\kappa U_n\unab_0^n\right]\right)\Phi(\ux) = \sue q(\ux),
\nal
where $W_{n}=V_n-U_n$.
If we discard the terms of $O(\ep^{2n})$ in this equation, we obtain a partial differential equation for $\Phi(\ux)$ of order $2n$.
We will use this approach to explicitly derive the nonclassical SP$_1$, SP$_2$, and SP$_3$ equations.
Higher-order equations can be derived from \cref{eq8} by continuing to follow the same procedure.

\subsection{Nonclassical diffusion equation (nonclassical SP$_1$)}
We discard the terms of $O(\ep^2)$ in \cref{eq8} and rewrite the equation as 
\bala%\label{2.20}
W_1\unab_0\Phi(\ux) + \kappa\Phi(\ux) = \sue q(\ux).
\nala
Multiplying this equation by $\ep$ and reverting to the original unscaled parameters, we obtain
\bal\label{eq9}
- \f{1}{6}\f{\sd}{\su}\unab^2\Phi(\ux) + \f{1-c}{\su}\Phi(\ux) = Q(\ux),
\nal
which is the nonclassical diffusion equation as derived in \cite{larvas11}. 

\subsection{Nonclassical simplified P$_2$ equation}
Discarding the terms of $O(\ep^4)$ in \cref{eq8} and rearranging the terms, we get
\bala%\label{2.24}
-\left(I + \ep^2\f{W_2\unab_0 +\kappa U_1}{W_1}\right)W_1\unab_0\Phi(\ux) = \kappa\Phi(\ux) - \sue q(\ux).
\nala
Operating on this equation by $\left(I - \ep^2\left[W_2\unab_0+\kappa U_1\right]/W_1\right)$ and discarding terms of $O(\ep^4)$, it becomes
\bala%\label{2.25}
W_1\unab_0& \left[\Phi(\ux)-\ep^2\f{W_2}{W_1^2}\left[\kappa\Phi(\ux) - \sue q(\ux)\right]\right] + \\
&\qquad \kappa \left[1-\ep^2\kappa\f{U_1}{W_1}\right]\Phi(\ux) =
\left[1-\ep^2\kappa\f{U_1}{W_1}\right]\sue q(\ux).
\nonumber
\nala
Finally, we multiply this equation by $\ep$ and revert to the original unscaled parameters to obtain the nonclassical SP$_2$ equation
\bal\label{eq10}
-\f{1}{6}\f{\sd}{\su}\unab^2 &\bigg[\Phi(\ux)+\lambda_1\left[(1-c)\Phi(\ux) - \su Q(\ux)\right]\bigg]+\\
 &\f{1-c}{\su} \big[1-\beta_1(1-c)\big]\Phi(\ux) =
\big[1-\beta_1(1-c)\big]Q(\ux),\nonumber
\nal
with $\lambda_1 = \f{3}{10}\f{\sq}{\sd^2} - \f{1}{3}\f{\st}{\su\sd}$ and 
$\beta_1 = \f{1}{3}\f{\st}{\su\sd} - 1$.

\subsection{Nonclassical simplified P$_3$ equations}
Discarding the terms of $O(\ep^6)$ in \cref{eq8}, we have
\bala
&\left(W_1\unab_0 + \ep^2\left[W_2\unab_0^2 + \kappa U_1\unab_0\right]+ \right.\\
&\qquad  \left. \ep^4\left[W_3\unab_0^3 + \kappa U_2\unab_0^2\right]\right)\Phi(\ux) + \kappa\Phi(\ux) = \sue q(\ux).\nonumber
\nala
We define
$
\nu(\ux) = \left(I + \ep^2\f{W_3\unab_0+\kappa U_2}{W_2}\right)\f{\ep^2}{2}\f{W_2}{W_1}\unab_0\Phi(\ux),\nonumber
$
and rewrite this equation as
\bala%\label{2.29}
W_1\unab_0\left[\Phi(\ux) + 2\nu(\ux) + \ep^2\kappa \f{U_1}{W_1}\Phi(\ux)\right] + \kappa\Phi(\ux) = \sue q(\ux).
\nala
Operating on $\nu(\ux)$ by $\left(I - \ep^2[W_3\unab_0+\kappa U_2]/W_2\right)$ and discarding terms of $O(\ep^6)$, we get
\bala%\label{2.31}
\ep^2\f{W_1}{\sue}\unab_0 &\left[\f{W_3}{W_1W_2}\nu(\ux)+\f{1}{2}\f{W_2}{W_1^2}\Phi(\ux)\right]+\\
&\qquad\qquad \f{1}{\sue}\left[1-\ep^2\kappa\f{U_2}{W_2}\right]\nu(\ux) = 0.
\nala

Finally, we multiply the two equations above by $\ep$ and $1/\ep$, respectively, and revert to the original unscaled parameters to obtain the nonclassical SP$_3$ equations
\bsub\label[pluraleq]{eq11}
\bal
&-\f{1}{6}\f{\sd}{\su}\unab^2\bigg[\big[1+\beta_1(1-c)\big]\Phi(\ux) + 2\nu(\ux)\bigg] + \label{eq11a}\\
&\hspace{15em}\f{1-c}{\su}\Phi(\ux) = Q(\ux),\nonumber\\
&-\f{1}{6}\f{\sd}{\su}\unab^2\left[\f{\lambda_1}{2}\Phi(\ux)+\lambda_2\nu(\ux)\right]+\f{1-\beta_2(1-c)}{\su}\nu(\ux) = 0,\label{eq11b}
\nal
\nsub
where $\lambda_1$ and $\beta_1$ are the same as in the SP$_2$ equations, and
\bala
\lambda_2 =& \f{1}{10\sd\st-9\su\sq }\times\\
&\quad \left[\f{9}{5}\sc - \f{27}{21}\f{\su\ss}{\sd} + 3\f{\st\sq}{\sd} - \f{10}{3} \f{\st^2}{\su}\right],\\
\beta_2 =& \f{1}{10\sde\ste-9\sue\sqe}\left[\f{10}{3}\f{\ste^2}{\sue} - \f{9}{5}\sce\right]-1.
\nala

%%%%%%%%%%%%%%%%%%%%%%%%%%%%%%%%%%%%%%%%%%%%%%%%%%%%%%%%%%%%%%%%%%%%%%%%%%%%%%%%
\section{Classical SP$_N$ Equations}
If $\Sigma_t(s) = \Sigma_t$ is independent of $s$, the nonclassical SP$_N$ equations derived in the previous section reduce to the SP$_N$ approximations to the classical transport equation \cite{mcl11}.
Under this assumption, the free-path distribution $p(s)$ is an exponential and $\bl s^m\bg = m!\Sigma_t^{-m}$.

Introducing this result into the nonclassical diffusion approximation given by \cref{eq9}, one can easily see that it reduces to the classical diffusion equation
\bala
- \f{1}{3\Sigma_t}\unab^2\Phi(\ux) + \Sigma_a\Phi(\ux) = Q(\ux).
\nala
We also obtain $\lambda_1 = \f{4}{5}$; $\lambda_2 = \f{11}{7}$; and $\beta_1 = \beta_2 = 0$.
In this case, the nonclassical SP$_2$ equation (\ref{eq10}) reduces to
\bala
- \f{1}{3\Sigma_t}\unab^2&\left[\Phi(\ux) + \f{4}{5}\f{\Sigma_a\Phi(\ux)- Q(\ux)}{\Sigma_t}\right] + \Sigma_a\Phi(\ux) = Q(\ux),
\nala
which is the classical SP$_2$ equation.
The nonclassical SP$_3$ equations (\ref{eq11}) reduce to 
\bsub
\bala
-\f{1}{3\Sigma_t}\unab^2\bigg[\Phi(\ux) + 2\nu(\ux)\bigg] + \Sigma_a\Phi(\ux) = Q(\ux),\\
-\f{1}{3\Sigma_t}\unab^2\left[\f{2}{5}\Phi(\ux)+\f{11}{7}\nu(\ux)\right]+\Sigma_t\nu(\ux) = 0,
\nala
\nsub
which are the classical SP$_3$ equations.

Moreover, $U_1=U_2=U_3...=0$ and $V_0 = 1$; $V_1 = -\f{1}{\sigma_t^2}$; $V_2 = -\f{4}{5\sigma_t^4}$; $V_3 = -\f{44}{35\sigma_t^6}$; with $\sigma_t = \Sigma_t/\ep$.
Thus, \cref{eq8} becomes
\bala
-\left(\f{1}{\sigma_t}\unab_0 + \f{4\ep^2}{5\sigma_t^3}\unab_0^2 + \f{44\ep^4}{35\sigma_t^5}\unab_0^3 + O(\ep^6)\right)\Phi(\ux) + \kappa\Phi(\ux) = q(\ux).
\nala
This is the asymptotic approximation to the classical transport equation with isotropic scattering that can be used to obtain the classical SP$_N$ equations \cite{mcl11}.  

\subsection{A comment on boundary conditions}
Our asymptotic analysis does not yield boundary conditions.
Therefore, in order to generate the numerical results in this paper, we manipulate the nonclassical SP$_N$ equations to a classical form with modified parameters, and then use classical (Marshak) vacuum boundary conditions.

Specifically, the nonclassical SP$_1$ equation with vacuum boundary conditions is written as
\bala
- \f{1}{3\wst}\unab^2\Phi(\ux) + \wsa\Phi(\ux) &= Q(\ux),\\
\f{1}{2}\Phi(\ux) -\f{1}{3\wst}\vec\un\cdot\unab\Phi(\ux) &= 0,
\nala
with $\wst = 2\f{\su}{\sd}$ and $\wsa = \f{1-c}{\su}$. Similarly, the nonclassical SP$_2$ equation with vacuum boundary conditions becomes
\bala
- \f{1}{3\wst}\unab^2\widehat\Phi(\ux) + \wsa\widehat\Phi(\ux) &= \wq(\ux),\\
\f{1}{2}\widehat\Phi(\ux) -\f{1}{3\wst}\vec\un\cdot\unab\widehat\Phi(\ux) &= 0,
\nala
with $\wst = 2\f{\su}{\sd}$; $\wsa =\f{(1-c)}{\su} \f{1-\beta_1(1-c)}{1+\lambda_1(1-c)}$; $\wq(\ux) = \f{1-\beta_1(1-c)}{1+\lambda_1(1-c)} Q(\ux)$; and $\Phi(\ux) = \frac{\widehat\Phi(\ux) + \lambda_1\su Q(\ux)}{1+\lambda_1(1-c)}$.

Finally, the nonclassical SP$_3$ equations with vacuum boundary conditions become
\bala
&-\f{1}{3\wst}\unab^2\big[\Phi(\ux) + 2\widehat\Phi_2(\ux)\big] + \wsa\Phi(\ux) = \wq(\ux),\\
&-\f{1}{3\wst}\unab^2\left[\f{2}{5}\Phi(\ux)+\left(\f{4}{5}+\f{27\wst}{35\widehat\Sigma_3}\right)\widehat\Phi_2(\ux)\right]+\widehat\Sigma_2\widehat\Phi_2(\ux) = 0,\\
&\f{1}{2}\Phi(\ux)-\f{1}{3\wst}\vec\un\cdot\unab\Phi(\ux) -\f{2}{3\wst}\vec\un\cdot\unab\widehat\Phi_2(\ux) +\f{5}{8}\widehat\Phi_2(\ux) = 0,\\
&-\f{1}{8}\Phi(\ux) +\f{5}{8}\widehat\Phi_2(\ux) -\f{3}{7\widehat\Sigma_3} \vec\un\cdot\unab\widehat\Phi_2(\ux) = 0,
\nala
with $\widehat\Phi_2(\ux) = \f{\nu(\ux)}{1+\beta_1(1-c)}$; $\wst = 2\f{\su}{\sd} $; $\wsa =\f{(1-c)}{\su} \f{1}{1+\beta_1(1-c)}$; $\wq(\ux) = \f{Q(\ux)}{1+\beta_1(1-c)}$; $\widehat\Sigma_2 = \f{4\left[1+\beta_1(1-c)\right]\left[1-\beta_2(1-c)\right]}{5\lambda_1\su}$; and $\widehat\Sigma_3 = \f{27}{28}\f{\lambda_1\wst}{\lambda_2\left[1+\beta_1(1-c)\right]-\lambda_1}$.

%%%%%%%%%%%%%%%%%%%%%%%%%%%%%%%%%%%%%%%%%%%%%%%%%%%%%%%%%%%%%%%%%%%%%%%%%%%%%%%%
\section{Numerical Results}
We consider transport taking place in a 1-D random periodic system similar to the one introduced in \cite{zuc94}, consisting of a random segment of periodically arranged solid and void layers of equal width $\ell=0.5$.
The total width of the system is given by $2X = 4\ell M$, where the integer $M$ (the total length of each material in the system) satisfies $M = \varepsilon^{-1}$.
Vacuum boundary conditions are assigned at $x=\pm X$. 

Cross sections and source in the void material are 0; the parameters of the solid material are $\Sigma_{t1} = 1$ and $Q_1 = 0.2/M^2$.
The absorption ratio is given by $1-c = 0.1/M^2$.
As $M$ increases, $\varepsilon$ decreases, and the 1-D system approaches the diffusive limit described in the asymptotic analysis.

To generate benchmark results for comparison, we used the same procedure presented in \cite{vas16}.
In this procedure, we obtain a physical realization of the system by choosing a continuous segment of two full layers (one of each material) and randomly placing the coordinate $x = 0$ in this segment.
Given this fixed realization of the system, the cross sections and source are now deterministic functions of space.

We solve the transport equation numerically for this realization using (i) the standard discrete ordinate method with a 16-point Gauss-Legendre quadrature set (S$_{16}$); and (ii) diamond differencing for the spatial discretization.
This procedure is repeated for different realizations of the random system.
Finally, we calculate the ensemble-averaged scalar flux by averaging the resulting scalar fluxes over all physical realizations.
\begin{figure}[ht] % replace 't' with 'b' to force it to be on the bottom
  \centering
  \includegraphics[scale=0.75]{fig_spn}
  \caption{Error of the nonclassical SP$_N$ estimates for the scalar flux with respect to the benchmark solution at $x=0$.}
  \label{fig}
\end{figure}

We calculate the scalar flux at the center of the system ($x=0$) for different values of $M$. 
The expectation is that the estimates obtained with the nonclassical SP$_N$ approximations will increase in accuracy as $M$ increases and $\ep\rightarrow 0$.

\Cref{fig} shows the percent relative error of the nonclassical SP$_N$ estimates with respect to the benchmark result.
As anticipated, the numerical results confirm the asymptotic analysis: (i) the accuracy of the SP$_N$ equations increases as $N$ increases; and (ii) for each choice of $N$, the error decreases as $M$ increases and the system approaches the diffusive limit.


%%%%%%%%%%%%%%%%%%%%%%%%%%%%%%%%%%%%%%%%%%%%%%%%%%%%%%%%%%%%%%%%%%%%%%%%%%%%%%%%
\section{Conclusions}
This paper derives a set of diffusion approximations to the nonclassical transport equation with isotropic scattering through an asymptotic analysis.
These approximations reduce to the simplified P$_N$ equations under the assumption of classical transport, and for that reason are labeled \textit{nonclassical} SP$_N$ equations.
Explicit equations are given for nonclassical SP$_1$ (diffusion), SP$_2$, and SP$_3$; higher-order equations can be derived by continuing to follow the same procedure.

Numerical results for a 1-D random periodic system are presented, validating the theoretical predictions.
Since the analysis does not yield boundary conditions, these numerical results were obtained by manipulating the nonclassical SP$_N$ equations into a structure that allows us to use classical boundary conditions.
This result paves the road to a more complete understanding of the diffusive behavior of the nonclassical transport theory.

%%%%%%%%%%%%%%%%%%%%%%%%%%%%%%%%%%%%%%%%%%%%%%%%%%%%%%%%%%%%%%%%%%%%%%%%%%%%%%%%
\section{Acknowledgments}
This paper was prepared by Richard Vasques and Rachel N. Slaybaugh under award number NRC-HQ-84-14-G-0052 from the Nuclear Regulatory Commission.
The statements, findings, conclusions, and recommendations are those of the authors and do not necessarily reflect the view of the U.S. Nuclear Regulatory Commission.

%%%%%%%%%%%%%%%%%%%%%%%%%%%%%%%%%%%%%%%%%%%%%%%%%%%%%%%%%%%%%%%%%%%%%%%%%%%%%%%%
\bibliographystyle{ans}
\bibliography{bibliography}
\end{document}


\documentclass[12pt]{article}

\usepackage[T1]{fontenc}
\usepackage[utf8]{inputenc}
\usepackage{lmodern}
\usepackage{amsthm}
\usepackage{amssymb}
\usepackage{amstext}
\usepackage{bm}
\usepackage{graphicx}
\usepackage{epstopdf}
\usepackage[export]{adjustbox}
\usepackage{subcaption}
\usepackage[breaklinks=true]{hyperref}
\usepackage{amsmath}
\usepackage{setspace}
\usepackage{morefloats} 
\usepackage{natbib}
\usepackage{fancyhdr}
\usepackage[capitalize]{cleveref}

% cleveref package
\crefalias{subequation}{equation}
\crefformat{pluraleq}{Eqs.~(#2#1#3)}

\pagestyle{plain}
% Margins
\renewcommand{\baselinestretch}{1.0} 
\setlength{\topmargin}{-0.5in} 
\setlength{\oddsidemargin}{0.25in} 
\setlength{\evensidemargin}{0.25in} 
\setlength{\textwidth}{6.0in} 
\setlength{\textheight}{9.0in} 
\setlength{\parskip}{0pt}    

% Labels of tables, sections, equations
\renewcommand\thetable{\Roman{table}}
\renewcommand{\thesection}{\arabic{section}.} 
\renewcommand{\thesubsection}{\thesection\arabic{subsection}.}
\renewcommand{\theequation}{\arabic{equation}}

%subsubfigures
\makeatletter
\newcounter{parentsubcaption}
\newenvironment{subsubcaption}
 {
 \renewcommand{\thesubfigure}{\roman{subfigure}}
 \refstepcounter{sub\@captype}%
  \protected@edef\theparentsubcaption{\@nameuse{thesub\@captype}}%
  \setcounter{parentsubcaption}{\value{sub\@captype}}%
  \setcounter{sub\@captype}{0}%
  \@namedef{thesub\@captype}{\alph{sub\@captype}.\theparentsubcaption}%
  \ignorespaces
}{%
  \setcounter{sub\@captype}{\value{parentsubcaption}}%
  \ignorespacesafterend
}
\makeatother


% hyperlinks
\hypersetup{colorlinks=false,
  pdftitle={Nonclassical Particle Transport in 1-D Random Periodic Media},
  pdfauthor={Richard Vasques and Kai Krycki and Rachel Slaybaugh}
}

\newcommand{\bl}{\big<}
\newcommand{\bg}{\big>}
\newcommand{\R}{\mathbb{R}}
\newcommand{\eps}{\varepsilon}
\renewcommand{\vec}[1]{\mathbf{#1}}
\newcommand{\mat}[1]{\mathbf{#1}}
\newcommand{\ux}{{\bm x}}
\newcommand{\un}{{\bf n}}
\newcommand{\uomega}{{\bf \Omega}}
\newcommand{\unabla}{{\bf \nabla}}
\newcommand{\ul}{\underline}
\newcommand{\seta}{\mathcal{A}}
\newcommand{\setb}{\mathcal{B}}
 \newcommand{\Keywords}[1]{\vspace{12pt}\par\noindent
{\small{\bf Keywords\/}: #1}}
 
\begin{document}

\title{Boundary Problems in 1-D Nonclassical Transport}

\author{{\bf R.\ Vasques $^{a,}$}\footnote{Email: \texttt{richard.vasques@fulbrightmail.org}; Phone: (510)-340-0930; Fax: (510)-643-9685} , {\bf K.\ Krycki $^{b}$} , {\bf R.N.\ Slaybaugh $^{a}$}\\ \\
\em {\bf $^a$}University of California, Berkeley\\
\em Department of Nuclear Engineering\\
\em 4155 Etcheverry Hall, Berkeley, CA 94720-1730\\
\and \\
\em {\bf $^b$}Aachen Institute for Nuclear Training GmbH \\
\em Jesuitenstraße 4, 52062 Aachen, Germany}
\date{}

\rhead{\runtitle}
\lhead{}
\pagestyle{fancy}
\makeatletter
\let\runtitle\@title
\makeatother

\maketitle

\begin{abstract}

TO BE DONE - 100 words max

\Keywords{keywords}
\end{abstract}

\doublespacing


%%%%%%%%%%%%%%%%%%%%%%%%%%%%%%%%%%%%%%%%%%%%%%%%%%%%%%%%%%%%%%%%%%%%%
\section{Introduction}\label{sec1}

The nonclassical theory has been introduced to address particle transport problems in which the particle flux experiences a nonexponential
attenuation law \citep{lar07,fra10,larvas11,vaslar14a}. This nonexponential behavior arises in certain inhomogeneous random media in which the locations of the scattering centers are spatially correlated; applications include neutron transport in Pebble Bed Reactors \citep{vaslar09,vas13,vaslar14b} and photon transport in atmospheric clouds \citep{kry13}.

The accuracy of the nonclassical transport equation in 1-D random periodic systems with an internal source was recently investigated \citep{mc15, nse16}.
It was found that the nonclassical theory greatly outperforms the atomic mix model in estimating the ensemble-averaged scalar flux in these systems.

In this work we analyze the performance of the nonclassical model in estimating the ensemble-averaged particle flux in 1-D boundary problems.
Specifically, we focus on the transmission problem in 1-D random periodic systems.
This paper is an expanded version of a recent work presented at the 24$^{\text{th}}$ International Conference on Transport Theory, in Taormina, Italy (Sep. 7-11, 2015).

The remainder of this paper is organized as follows. 

...

...

...

\section{The Nonclassical Transport Equation}\label{sec2}

Let the path-length $s$ be defined as
\begin{align}\label{eq1}
s =  \begin{array}{l}
\text{ the path-length traveled by the particle since}\vspace{-10pt}\\
\text{ its previous interaction (birth or scattering).}\\
\end{array}
\end{align}
Then, for the case of particle transport with coherent isotropic scattering, the nonclassical linear Boltzmann equation is writen as \citep{vaslar14a}
\begin{align}\label{eq2}
\frac{\partial\psi}{\partial s}(\ux,\uomega,s) + \uomega\cdot\bm\nabla & \psi(\ux,\uomega,s) + \Sigma_t(\uomega,s)\psi(\ux,\uomega,s) 
\\&= \delta(s)\left[ \frac{c}{4\pi}\int_{4\pi}\int_0^\infty \Sigma_t(\uomega',s')\psi(\ux,\uomega',s')ds' d\Omega' + Q(\ux,\uomega) \right]\,, \nonumber
\end{align}
where $\ux = (x,y,z)=$ position, $\uomega = (\Omega_x,\Omega_y,\Omega_z)=$ direction of flight ($|\uomega|=1$), $\psi$ is the nonclassical angular flux, $c$ is the scattering ratio (and the scattering cross section $\Sigma_s= c\Sigma_t$), and $Q$ is the source.
The probability of collision $\Sigma_t(\uomega,s)ds$ is defined as
\begin{align}\label{eq3}
\Sigma_t(\uomega,s)ds =  \begin{array}{l}
\text{ the probability (ensemble-averaged over all physical}\vspace{-10pt}\\
\text{ realizations) that a particle, scattered or born at any}\vspace{-10pt}\\
\text{ point $\ux$ and traveling in the direction $\uomega$, will experience}\vspace{-10pt}\\
\text{ a collision between $\ux + s\uomega$ and $\ux + (s+ds)\uomega$.}
\end{array}
\end{align}
The path-length distribution function is given by
\begin{subequations}\label[pluraleq]{eq4}
\begin{align}\label{eq4a}
p(\uomega,s) = \Sigma_t(\uomega,s)\exp\left( -\int_0^s\Sigma_t(\uomega,s')ds'\right),
\end{align}
such that
\begin{align}\label{eq4b}
\Sigma_t(\uomega,s)=\frac{p(\uomega,s)}{1-\int_0^sp(\uomega,s')ds'}.
\end{align}
\end{subequations}

For the 1-D systems considered in this paper, \cref{eq2} is written as
\begin{align}\label{eq5}
\frac{\partial\psi}{\partial s}(x,\mu,s) + \mu&\frac{\partial \psi}{\partial x}(x,\mu,s) + \Sigma_t(\mu,s)\psi(x,\mu,s) 
\\& = \delta(s)\left[ \frac{c}{2}\int_{-1}^1\int_0^\infty \Sigma_t(\mu',s')\psi(x,\mu',s')ds' d\mu' + Q(x,\mu)\right]\,. \nonumber
\end{align}
If classical transport takes place, $\Sigma_t(\mu,s) = \Sigma_t =$ cte, the path-length distribution reduces to the exponential form
\begin{align}\label{eq6}
	p(s) = \Sigma_t e^{-\Sigma_t s},
\end{align}
and \cref{eq2} becomes the classical linear Boltzmann equation
\begin{subequations}\label[pluraleq]{eq7}
\begin{align}\label{eq7a}
\mu\frac{\partial\Psi}{\partial x}(x,\mu) + \Sigma_t\Psi(x,\mu) = \frac{\Sigma_s}{2}\int_{-1}^1\Psi(x,\mu')d\mu' +  Q(x,\mu) \,
\end{align}
for the classical angular flux 
\begin{align}
\Psi(x,\mu) = \int_0^\infty \psi(x,\mu,s)ds.
\end{align} 
\end{subequations}

...

...

\section{The 1-D Random Periodic System}\label{sec3}

We consider the 1-D physical system introduced by \cite{zuc94} and described in detail in \cite{nse16}.
This system consists of alternating layers of periodically arranged materials with period $\ell = \ell_1 + \ell_2$, where $\ell_i$ represents the length of each layer of material $i \in \{1,2\}$.
A sketch of the periodic system is given in \cref{fig1}.

The random quality of this system arises from randomly placing the periodic arrangement in the $x$-axis.
As an example, a single physical realization of this system can be obtained by choosing a continuous segment of two full layers (one of each material) and randomly placing the coordinate $x=0$ in this segment.
Thus, the cross sections and source are {\em stochastic} functions of space, such that if $x$ is in material $i$:
\begin{subequations}\label[pluraleq]{eq8}
\begin{align}
\Sigma_t(x) &= \Sigma_{ti}\, ,\\
\Sigma_s(x) &= c_i \Sigma_{ti}\, ,\\
Q(x,\mu) &= Q_i(x,\mu) \, ,
\end{align}
\end{subequations}
where $\Sigma_{ti}$, $c_i$, and $Q_i$ represent the total cross section, scattering ratio, and source in material $i$. 

For this class of 1-D problems, an analytical expression for the distribution function $p(\mu,s)$ of a particle's distance-to-collision in the direction $\mu$ can be obtained.
Then, using the identity in \cref{eq4b} for this 1-D case, one can solve \cref{eq5}.

\section{The Path-length Distribution Function}\label{sec4}

Let us examine a given realization of the 1-D system described in \cref{sec3}.
Without loss of generality, consider a particle born or scattered in material $i$ at a horizontal distance $x_0$ of the next intersection, with direction of flight $\mu$. Following \cite{nse16}, the probability that this particle will experience its first collision while traveling a distance between $s$ and $s+ds$ is given by
\begin{align}\label{eq9}
\hat p_i(x_0,\mu, s)ds = ds\left\{
\begin{array}{ll}
\Sigma_{ti}e^{-\Sigma_{ti}s}, & \text{if } 0\leq s|\mu|\leq x_0  \\
\Sigma_{tj}e^{-\Sigma_{tj}s-(\Sigma_{ti}-\Sigma_{tj})(x_0+n\ell_i)/|\mu|}, & \text{if } x_0+n\ell< s|\mu|\leq  x_0+n\ell+\ell_j \\
\Sigma_{ti}e^{-\Sigma_{ti}s-(\Sigma_{tj}-\Sigma_{ti})(n+1)\ell_j/|\mu|}, & \\
 & \hspace{-1.1cm}\text{if } x_0+n\ell+\ell_j< s|\mu|\leq x_0+(n+1)\ell
\end{array}
\right.
\end{align}
where $n=0, 1, 2, ...$; $i,j \in\{1,2\}$; $i\neq j$; and $\ell = \ell_i+\ell_j$.

Considering all possible realizations, the \textit{ensemble-averaged} path-length distribution function of particles born or scattered in material $i$ with direction of flight $\mu$ is given by
\begin{align}\label{eq10}
p_i(\mu,s) = \frac{1}{\ell_i}\int_0^{\ell_i} \hat p_i(x_0,\mu,s) dx_0.
\end{align}
Therefore, the ensemble-averaged path-length distribution function for particles with direction of flight $\mu$ born (or scattered) anywhere in the 1-D random periodic system is given by the weighted average
\begin{align}\label{eq11}
p(\mu,s) &= \lambda_1 p_1(\mu,s) + \lambda_2p_2(\mu,s),
\end{align}
where $\lambda_i$ is the probability that any given birth or scattering event takes place in material $i$.
As expected, if $\Sigma_{t1}=\Sigma_{t2}=\Sigma_t$, \cref{eq9,eq10,eq11} yield the exponential path-length distribution given in \cref{eq6}.

\subsection{Solid-Void Medium}\label{sec41}

The numerical results included in this paper are for solid-void systems, with material 2 being void ($\Sigma_{t2}=Q_2=0$). Thus, as shown in \cite{nse16}, \cref{eq9,eq10} yield the following expressions for $p_1(\mu,s)$:
\begin{subequations}\label[pluraleq]{eq12}
\begin{itemize}
\item Case 1: $\ell_1<\ell_2$
\end{itemize}
\begin{align}
p_1(\mu,s) = \left\{
\begin{array}{ll}
\frac{\Sigma_{t1}}{\ell_1}(n\ell +\ell_1-s|\mu|)e^{-\Sigma_{t1}(s-n\ell_2/|\mu|)}, & \text{if } n\ell\leq s|\mu| \leq n\ell+\ell_1\\
0, & \text{if } n\ell+\ell_1 \leq s|\mu| \leq n\ell+\ell_2\\
\frac{\Sigma_{t1}}{\ell_1}(s|\mu|-n\ell-\ell_2)e^{-\Sigma_{t1}[s-(n+1)\ell_2/|\mu|]}, & \text{if } n\ell+\ell_2 \leq s|\mu| \leq (n+1)\ell\\
\end{array}
\right.
\end{align}
\begin{itemize}
\item Case 2: $\ell_1=\ell_2$
\end{itemize}
\begin{align}
p_1(\mu,s) = \left\{
\begin{array}{ll}
\frac{\Sigma_{t1}}{\ell_1}(n\ell +\ell_1-s|\mu|)e^{-\Sigma_{t1}(s-n\ell_2/|\mu|)}, & \text{if } n\ell\leq s|\mu| \leq n\ell+\ell_1\\
\frac{\Sigma_{t1}}{\ell_1}(s|\mu|-n\ell-\ell_2)e^{-\Sigma_{t1}[s-(n+1)\ell_2/|\mu|]}, & \text{if } n\ell+\ell_2 \leq s|\mu| \leq (n+1)\ell\\
\end{array}
\right.
\end{align}
\begin{itemize}
\item Case 3: $\ell_1>\ell_2$
\end{itemize}
\begin{align}
p_1(\mu,s) = \left\{
\begin{array}{ll}
\frac{\Sigma_{t1}}{\ell_1}(n\ell +\ell_1-s|\mu|)e^{-\Sigma_{t1}(s-n\ell_2/|\mu|)}, & \\
\hspace{6cm}\text{if } n\ell\leq s|\mu| \leq n\ell+\ell_2 & \\
\frac{\Sigma_{t1}}{\ell_1}[(n\ell +\ell_2-s|\mu|)(1-e^{\Sigma_{t1}\ell_2/|\mu|}) +\ell_1-\ell_2]e^{-\Sigma_{t1}(s-n\ell_2/|\mu|)}, & \\
\hspace{6cm}\text{if } n\ell+\ell_2\leq s|\mu| \leq n\ell+\ell_1 & \\
\frac{\Sigma_{t1}}{\ell_1}(s|\mu|-n\ell-\ell_2)e^{-\Sigma_{t1}[s-(n+1)\ell_2/|\mu|]}, & \\
\hspace{6cm}\text{if } n\ell+\ell_1 \leq s|\mu| \leq (n+1)\ell & 
\end{array}
\right.
\end{align}
\end{subequations}
where $n=0, 1, 2, ...$ .

Particles cannot scatter or be born in material 2; they can, however, enter the system through the boundaries. If the material located at the boundary through which a particle enters is void, \cref{eq12} does not apply. The ensemble-averaged path-length distribution function of particles entering the system through a void layer is given by
\begin{subequations}\label[pluraleq]{eq13}
\begin{itemize}
\item Case 1: $\ell_1<\ell_2$
\end{itemize}
\begin{align}
p_2(\mu,s) = \left\{
\begin{array}{ll}
\frac{|\mu|}{\ell_2}\left(e^{-\Sigma_{t1}n\ell_1/|\mu|}-e^{-\Sigma_{t1}(s-n\ell_2/|\mu|)}\right), & \text{if } n\ell\leq s|\mu| \leq n\ell+\ell_1\\
\frac{|\mu|}{\ell_2}e^{-\Sigma_{t1}n\ell_1/|\mu|}\left(1-e^{-\Sigma_{t1}\ell_1/|\mu|}\right), & \text{if } n\ell+\ell_1 \leq s|\mu| \leq n\ell+\ell_2\\
\frac{|\mu|}{\ell_2}\left(e^{-\Sigma_{t1}[s-(n+1)\ell_2/|\mu|]}-e^{-\Sigma_{t1}(n+1)\ell_1/|\mu|}\right), & \text{if } n\ell+\ell_2 \leq s|\mu| \leq (n+1)\ell\\
\end{array}
\right.
\end{align}
\begin{itemize}
\item Case 2: $\ell_1=\ell_2$
\end{itemize}
\begin{align}
p_2(\mu,s) = \left\{
\begin{array}{ll}
\frac{|\mu|}{\ell_2}\left(e^{-\Sigma_{t1}n\ell_1/|\mu|}-e^{-\Sigma_{t1}(s-n\ell_2/|\mu|)}\right), & \text{if } n\ell\leq s|\mu| \leq n\ell+\ell_1\\
\frac{|\mu|}{\ell_2}\left(e^{-\Sigma_{t1}[s-(n+1)\ell_2/|\mu|]}-e^{-\Sigma_{t1}(n+1)\ell_1/|\mu|}\right), & \text{if } n\ell+\ell_2 \leq s|\mu| \leq (n+1)\ell\\
\end{array}
\right.
\end{align}
\begin{itemize}
\item Case 3: $\ell_1>\ell_2$
\end{itemize}
\begin{align}
p_2(\mu,s) = \left\{
\begin{array}{ll}
\frac{|\mu|}{\ell_2}\left(e^{-\Sigma_{t1}n\ell_1/|\mu|}-e^{-\Sigma_{t1}(s-n\ell_2/|\mu|)}\right), & \text{if } n\ell\leq s|\mu| \leq n\ell+\ell_2\\
\frac{|\mu|}{\ell_2}e^{-\Sigma_{t1}(s-n\ell_2/|\mu|)}\left(e^{\Sigma_{t1}\ell_2/|\mu|}-1\right), & \text{if } n\ell+\ell_2 \leq s|\mu| \leq n\ell+\ell_1\\
\frac{|\mu|}{\ell_2}\left(e^{-\Sigma_{t1}[s-(n+1)\ell_2/|\mu|]}-e^{-\Sigma_{t1}(n+1)\ell_1/|\mu|}\right), & \text{if } n\ell+\ell_1 \leq s|\mu| \leq (n+1)\ell\\
\end{array}
\right.
\end{align}
\end{subequations}
where $n=0, 1, 2, ...$ .

...

...

...




%%%%%%%%%%%%%%%%%%%%%%%%%%%%%%%%%%%%%%%%%%%%%%%%%%%%%%%%%%%%%%%%%%%%%
\section*{Acknowledgments}

This paper was prepared by Richard Vasques and Rachel Slaybaugh under award number NRC-HQ-84-14-G-0052 from the Nuclear Regulatory Commission.
The statements, findings, conclusions, and recommendations are those of the authors and do not necessarily reflect the view of the U.S.\ Nuclear Regulatory Commission.

%%%%%%%%%%%%%%%%%%%%%%%%%%%%%%%%%%%%%%%%%%%%%%%%%%%%%%%%%%%%%%%%%%%%%
\setlength{\baselineskip}{12pt}

\begin{thebibliography}{}

\bibitem[Frank and Goudon, 2010]{fra10}
Frank, M., Goudon, T. (2010). On a generalized Boltzmann equation for non-classical particle transport. {\it Kin. Rel. Models}, {\bf 3}:395--407.

\bibitem[Krycki et al., 2013]{kry13}
Krycki, K., Turpault, R., Frank, M., Berthon, C. (2013). Asymptotic preserving numerical schemes for a non-classical radiation transport model for atmospheric clouds. {\it Math. Meth. Appl. Sci.}, {\bf 36}:2101--2116.

\bibitem[Larsen, 2007]{lar07}
Larsen, E.W. (2007). A generalized Boltzmann equation for non-classical particle transport. {\it
Proc. International Topical Meeting on Mathematics \& Computation and Supercomputing in Nuclear Applications,
M\&C+SNA 2007}, Monterey, CA, U.S.A., Apr. 15-19, on CD-ROM.

\bibitem[Larsen and Vasques, 2011]{larvas11}
Larsen, E.W., Vasques, R. (2011). A generalized linear Boltzmann equation for non-classical
particle transport. {\it J. Quant. Spectrosc. Radiat. Transfer}, {\bf 112}:619--631.

\bibitem[Vasques and Larsen, 2009]{vaslar09}
Vasques, R., Larsen, E.W. (2009). Anisotropic diffusion in model 2-D pebble-bed reactor cores. {\it
Proc. International Conference on Mathematics, Computational Methods \& Reactor Physics, M\&C 2009},
Saratoga Springs, NY, U.S.A., May 3-7, on CD-ROM.

\bibitem[Vasques, 2013]{vas13}
Vasques, R. (2013). Estimating anisotropic diffusion of neutrons near the boundary of a pebble bed random system. {\it Proc. International Conference on Mathematics and Computational Methods Applied to Nuclear Science \& Engineering, M\&C 2013}, Sun Valley, ID, U.S.A., May 5-9, pp. 1736--47.

\bibitem[Vasques and Larsen, 2014a]{vaslar14a}
Vasques, R., Larsen, E.W. (2014a). Non-classical particle transport with angular-dependent path-length
distributions. I: Theory. {\it Ann. Nucl. Energy}, {\bf 70}:292--300.

\bibitem[Vasques and Larsen, 2014b]{vaslar14b}
Vasques, R., Larsen, E.W. (2014b). Non-classical particle transport with angular-dependent path-length
distributions. II: Application to pebble bed reactor cores. {\it Ann. Nucl. Energy}, 
{\bf 70}:301--311.

\bibitem[Vasques and Krycki, 2015]{mc15}
Vasques, R., Krycki, K. (2015). On the accuracy of the non-classical transport equation in 1-D random periodic media. {\it Proc. Joint International Conference on Mathematics and Computation, Supercomputing in Nuclear Applications and the Monte Carlo Method, M\&C+SNA+MC 2015}, Nashville, TN, U.S.A., Apr. 19-23, on CD-ROM.

\bibitem[Vasques et al., 2016]{nse16}
Vasques, R., Krycki, K., Slaybaugh, R.N. (2016). Nonclassical Particle Transport in 1-D Random Periodic Media. {\it ArXiv e-prints}, {\bf arxiv:1602.00825} [nucl-th].

\bibitem[Zuchuat et al., 1994]{zuc94}
Zuchuat, O., Sanchez, R., Zmijarevic, I., Malvagi, F. (1994). Transport in renewal statistical media: benchmarking and comparison with models. {\it J. Quant. Spectrosc. Radiat. Transfer}, {\bf 51}:689--722.

\end{thebibliography}

\pagebreak

\begin{figure}[htb]
  \centering
%  \includegraphics[width=\textwidth]{fig1.eps}
  \caption{A sketch of the periodic medium}
  \label{fig1}
\end{figure}

\end{document}


